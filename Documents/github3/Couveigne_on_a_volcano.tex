\documentclass[10pt,a4paper]{article}
\usepackage[utf8]{inputenc}
\usepackage[english]{babel}
\usepackage[T1]{fontenc}
\usepackage{hyperref}
\usepackage[english]{babel}
\usepackage{amssymb,amsthm,amsmath,amsfonts}
\usepackage{units}
\usepackage{lmodern}
\usepackage{graphicx}
\bibliographystyle{plain}
\usepackage{epsfig}
\usepackage{epstopdf}
\usepackage{makeidx}
\usepackage{gloss}
\usepackage{xcolor}
\usepackage{algorithmic}
\usepackage{algorithm}
\def\glossname{Glossaire}
\usepackage{tikz}
\usepackage{fancyhdr}
\pagestyle{fancy}

\theoremstyle{plain}
\newtheorem{thm}{Theoreme}
\theoremstyle{definition} 
\newtheorem{lem}[thm]{Lemma}
\theoremstyle{definition} 
\newtheorem{cor}[thm]{Corollary}
\theoremstyle{definition} 
\newtheorem{prop}[thm]{Proposition}
\theoremstyle{definition} 
\newtheorem{defi}[thm]{Definition}
\theoremstyle{remark} 
\newtheorem{rem}[thm]{Remark}
\theoremstyle{remark} 
\newtheorem{exe}[thm]{Example}
\author{Luca DeFeo,Cyril Hugounenq,}

\begin{document}
\section{Reminder on Couveignes's algorithm}

\input{Reminder_Couveignes}
\section{Reminder on isogeny volcano}
\begin{prop}
Let $E$ be an ordinary elliptic curve then it's endomorphism ring is an order in a quadratic field. We will denote $K$ the quadratic field and $\mathcal{O}$ the endomorphism ring associated to $E$ up to isomorphism.
\end{prop}

\begin{defi}
We will denote by $\mathcal{O}_K$ the algebraic integers of $K$, it's the maximal order in terms of inclusion. We will denote by $[1,\omega_K]$ a $\mathbb{Z}$ basis of $\mathcal{O}_K$
\end{defi}

\begin{defi}
We denote by $\pi$ the Frobenius endomorphism which sends a point of $E$ of coordinates $(x,y)$ on a point of $E$ of coordinate $(x^p,y^p)$.
\end{defi}

We then have $\mathbb{Z}[\pi] \subset \mathcal{O} \subset \mathcal{O}_K$.

Dire la structure de la courbe elliptique en haut du cratère, et parler des indices

\begin{prop} \label{structelevation}
Let $\mathbb{F}_q$ be a finite field, let $E(\mathbb{F}_q)[2^{\infty}]=\mathbb{Z}/2^{h+j}\mathbb{Z} \times \mathbb{Z}/2^{h}\mathbb{Z}$ with $ j \geqslant 0$ and $\nu_2(q)>h>1$ a curve on the crater, then $E(\mathbb{F}_{q^2})[2^{\infty}] \supset \mathbb{Z}/2^{h+j+1}\mathbb{Z} \times \mathbb{Z}/2^{h+1}\mathbb{Z}$
\end{prop}

\begin{proof}
See lemme 6.5.2 page 67  of Mireille Fouquet \cite{Fouquet01}, the case $l ||g$ with $l=2$ is not treated but it can be proved with an adapted proof of the one quoted here.
\end{proof}

\begin{rem}
The equality holds when we are working on a regular volcano.
\end{rem}

\section{Computing a canonical basis}
\input{Computing_Tate_Module}
\section{Interpolating the two basis}
\input{Interpolation}


\end{document}